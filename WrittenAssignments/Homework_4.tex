\documentclass[a4paper,12pt]{article}

%Eingabe deutscher Umlaute
\usepackage[utf8]{inputenc}
\usepackage[T1]{fontenc}
\usepackage[ngerman]{babel}

\usepackage[colorlinks=true,linkcolor=green]{hyperref}


%Mathematische Symbole
\usepackage{amsmath}
\usepackage{mathtools}
\usepackage{amsthm}
\newtheorem{theorem}{Theorem}
\usepackage{amssymb}
\usepackage{mathdots}
\usepackage{bbm}
\usepackage{textgreek}
\usepackage[linewidth=1pt]{mdframed}
\usepackage{blindtext}

\usepackage[pdftex]{graphicx}

 %Weniger breite Raender
 \usepackage{a4wide}
\usepackage[right = 2cm, top=2cm, bottom=2.5cm,left=2cm]{geometry} 


%Normale Papiereinstellungen; Kein Einzug bei neuem Absatz.
\pagestyle{plain}
\setlength\parindent{0pt}


%%%%%%%%%%%%%%%%%%%%%%%%%%
%%%%%%%%Abkuerzende Befehle%%%%%%%
%%%%%%%%%%%%%%%%%%%%%%%%%%
%Erstes Argument {} enthaelt jeweils die Abkuerzung, zweites Argument {} den Latex-Befehl

% Zahlbereiche
\newcommand{\IN}{\mathbb{N}}
\newcommand{\IQ}{\mathbb{Q}}
\newcommand{\IZ}{\mathbb{Z}}
\newcommand{\IR}{\mathbb{R}}
\newcommand{\IC}{\mathbb{C}}

%Wahrscheinlichkeit und Erwartungswert
\newcommand{\IP}{\mathbb{P}}
\newcommand{\IE}{\mathbb{E}}

%Indikatorfunktion
\newcommand{\Ii}{\mathbbm{1}}

%Abkuerzungen fuer Sigma-Algebren etc.
\newcommand{\sP}{\mathcal{P}}
\newcommand{\sA}{\mathcal{A}}
\newcommand{\sC}{\mathcal{C}}
\newcommand{\sX}{\mathcal{X}}
\newcommand{\sE}{\mathcal{E}}
\newcommand{\sU}{\mathcal{U}}
\newcommand{\sN}{\mathcal{N}}
\newcommand{\sB}{\mathcal{B}_{\IR}}

\newcommand{\seqA}{$(A_{i})_{i\in\IN}\,$}
\newcommand{\unionA}{$\cup_{i\in\IN}{A_{i}}\,$}

\newcommand{\sig}{\textsigma\,}
\newcommand{\omeg}{\textOmega\,}

\newcommand\defeq{\stackrel{\mathclap{\normalfont\mbox{def}}}{=}}
\newcommand\totprobeq{\stackrel{\mathclap{\normalfont\mbox{LoTP}}}{=}}
\newcommand\propaeq{\stackrel{\mathclap{\normalfont\mbox{(a)}}}{=}}
\newcommand\propbeq{\stackrel{\mathclap{\normalfont\mbox{(b)}}}{=}}

\newcommand{\expKX}{\IE(e^{kX})}
\newcommand{\expKY}{\IE(e^{kY})}
\newcommand{\expX}{\IE(X)}
\newcommand{\gammaF}{\frac{\lambda^{\alpha}}{\Gamma(\alpha)}}
\newcommand{\gammaDenom}{\frac{1}{\Gamma(\alpha)}}
\newcommand{\datFrac}{\left( \frac{1}{\lambda - k} \right)^{\alpha}}
\newcommand{\datFracL}{\left( \frac{\lambda}{\lambda - k} \right)^{\alpha}}
\newcommand{\kdermx}{\frac{dM_{X}}{dk}\big|_{k=0}}
\newcommand{\expint}[1]{\int_{0}^{\infty}{e^{#1 }dx}}

\makeatletter
\renewcommand*{\eqref}[1]{%
  \hyperref[{#1}]{\textup{\tagform@{\ref*{#1}}}}%
}
\makeatother

%Aufzaehlungen bei enumerate werden (a),(b),(c)
\renewcommand{\labelenumi}{(\alph{enumi})}

\title{
	Homework  \\
	\large Computational Statistics and Data Analysis \\
	\large Summer Semester, 2020
	}
\author{Ryan Hutchins \\ 
Ruprecht Karls Universit\"at Heidelberg}
\date{8 Mai, 2020}

\begin{document}
\maketitle

You may find the code for this assignment \href{https://github.com/GoliathMarks/Computational_Statistics/tree/master/CompStatsHomeworkFour}{here}.

\section{Problem 1: Numerical Optimization}

\begin{enumerate}
\item For this part, we implement the gradient descent. We used the closed form of the gradient (a 2-D vector) in our computions. This closed form is:
\begin{equation}
\nabla_{\theta}l_{X}(\theta) = \sum_{i=1}^{N}{\left[ \frac{ \exp(-\beta (\theta -x_{i})) } { 1 + \exp(-\beta (\theta -x_{i})) } - y_{i}  \right] \left( \theta - x_{i}, \beta  \right)  }
\end{equation}
The estimate that I got for running with minimum change between steps of 0.01 was $\beta = 2.2233, \theta =0.4413$.


\item For the Newton-Raphson method, we once again used the closed form of the second partial derivatives in the Hessian matrix. The Hessian matrix in our case is:
$$
\begin{bmatrix}
\frac{\partial^{2}l_{X}(\theta)}{\partial \beta^{2}} & \frac{\partial^{2}l_{X}(\theta)}{\partial \beta \partial \theta} \\
\frac{\partial^{2}l_{X}(\theta)}{\partial \theta \partial \beta} & \frac{\partial^{2}l_{X}(\theta)}{\partial \theta^{2}} \\
\end{bmatrix}
$$
and the closed forms of the derivatives were, skipping the steps (and there were a good few) of algebra required to get there:
\begin{equation}
\frac{\partial^{2}l_{X}(\theta)}{\partial \beta^{2}} = \sum_{i=1}^{N}{  \frac{ -(\theta - x_{i})^{2}  \exp(-\beta (\theta -x_{i}))   }{\left[  1 + \exp(-\beta (\theta -x_{i}))   \right]^{2} }  }
\end{equation}

\begin{equation}
\frac{\partial^{2}l_{X}(\theta)}{\partial \theta^{2}} = \sum_{i=1}^{N}{  \frac{  -\beta^{2} \exp(-\beta (\theta -x_{i})) }  {\left[  1 + \exp(-\beta (\theta -x_{i}))   \right]^{2} }  }
\end{equation}

\begin{equation}
\frac{\partial^{2}l_{X}(\theta)}{\partial \beta \partial \theta} = \frac{\partial^{2}l_{X}(\theta)}{\partial \theta \partial \beta} = \sum_{i = 1}^{N}{ \frac{  (1-\beta (\theta - x_{i})) \exp(-\beta (\theta -x_{i})) + \exp(-2\beta (\theta -x_{i}))   }    { \left[  1 + \exp(-\beta (\theta -x_{i}))   \right]^{2}  }    }
\end{equation}
and in this last case, I am very grateful for the equality of mixed partial derivatives. The Newton-Raphson method returned parameters estimates of $\beta = 3.1*10^-4, theta = -3.8865$. However, the probability that I made a mistake either in my code implementation or my pen-and-paper computation of the derivative is high. 

\end{enumerate}

\section{Problem 2}
\begin{enumerate}
\item Code on Github.
\item The null hypothesis, $H_{0}$ is that the social distancing measures have no effect and that the variation in infection rate is due to randomness. The alternative hypothesis is that social distancing measures are effective, i.e., that the chance of seeing a decrease in case rate between corresponding days before and after the measures were introduced is greater than $\frac{1}{2}$. 
\item Executing the test for Germany over the week starting March 9th and pairing the days up with their corresponding days over the week starting on March 23rd showed that the estimated new infection rate was less all seven days. The probability of this under the null hypothesis is roughly 0.008. We could conclude from this that social distancing measures are effective at reducing infection rates. However, it should be noted that there could be confounding factors that were not accounted for in our model. 
\end{enumerate}






\end{document}